% vim:ft=tex:
%
\documentclass[12pt]{article}

\title{
	State of the art
}
\author{
	Juan Francisco Martínez --- \texttt{juan.martinez@bsc.es}
}

\begin{document}
\maketitle

A good study of the state of the art of the research problem is crucial for the good development of the master' thesis. According to this webpage\footnote{\url{https://blog.babak.no/en/2007/05/22/why-and-how-to-write-the-state-of-the-art/}} the key reasons for put effort of SOTA\footnote{State of the art} are:

\begin{itemize}
    \item It can be learned a lot about the research problem from other authors that were trying to solve it.
    \item If you can do an state of the art study it would mean that the target of study have some relevance for the scientific community.
    \item It can show different approaches. By looking to other approaches you can evaluate your own one.
    \item Maybe some parts of the other researchers work can be reused in your own research.
\end{itemize}

I have gathered some articles related to my research problem. In the next section i am going to present a little review about every one of them.

\section{Paper reviews}
\subsection{Detecting patterns in MPI Communication Traces}

Hola manola

\end{document}
