\chapter{Results}

\section{Scalability}\label{s:scalability}

Clustering will not be done with all MPI events presented on the whole trace but
just with these events that results from a compression from the original
tracefile. This compression consists on the aggregation of information from 
different instances of the same MPI event like inter-arrival time, number of
instances, duration, \ldots that since applications used to present very repetitive 
behavior the compression ratios will presumably be high. There is not the
responsibility of this section to describe the algorithm followed to do so that
is going to be described in \ref{ss:trace_reduction} but to demonstrate by a
quantitative analysis that this assumption is true. The experiments were done
by extracting traces from NPB suite with different problem sizes in a weak
scaling fashion and figure out how many unique MPI events are retrieved 
in every case for the clustering phase.

\section{Validation}\label{s:validation}

Some results
